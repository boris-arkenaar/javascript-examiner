\documentclass{article}
\begin{document}
\textbf{Hoofdstuk 1: Inleiding}
\\*
\\*Geef een korte toelichting op de achtergronden van het projectplan:
\\*Wat ging eraan vooraf?
\\*Welke stappen zijn genomen?
\newline

\noindent
Geef vervolgens een toelichting op de opbouw van het projectplan als leeswijzer. Per hoofdstuk geeft u in één of twee zinnen aan wat er in dat betreffende hoofdstuk staat. Tot slot geeft u in de inleiding een voorstel tot uitvoering aan.
Dat wil zeggen:
Wanneer denkt men te starten met de uitvoering van het project?
Indien er tijd zit tussen de afronding van het projectplan en de start van het project, wat wordt er in de tussentijd gedaan?
\newline
\newline

Hoofdstuk 2: Uitgangssituatie
\\*
\\*Beantwoord de vragen:
\\*Waarom wil de opdrachtgever dit project?
\\*Wat is de huidige situatie?
\newline

\noindent
\textbf{Paragraaf 2.1 Aanleiding}
\\*Welke problemen en oorzaken zijn de aanleiding geweest tot de wens om te veranderen?
\\*Hoe ziet de omgeving eruit? Beschrijf deze vanuit het oogpunt van de opdrachtgever.
\newline

\noindent
\textbf{Paragraaf 2.2 Uitgangsdocumentatie}
\\*Welke documentatie ligt ten grondslag aan het project?
\\*Welke kwaliteiten heeft deze documentatie?
\\*Welke activiteiten moeten er nog verricht worden om deze documentatie te complementeren?
\newline
\newline

Hoofdstuk 3: Projectresultaat
\\*
\\*Geef antwoord op de vraag:
\\*Wat is het uieindelijke resultaat van het project?
\newline

\noindent
\textbf{Paragraaf 3.1 Doelstelling}
\\*Wat is het achterliggende doel van de opdrachtgever?
\\*Wat is de koppeling tussen bedrijfsprocessen en de omgeving, gezien door de bril van de opdrachtgever?
\newline

\noindent
\textbf{Paragraaf 3.2 Resultaat}
\\*Wat is aan het einde van het project het concrete resultaat?
\newline

\noindent
\textbf{Paragraaf 3.3 Kwaliteit}
\\*Wat zijn de kaliteitseisen die gesteld worden aan het eindresultaat?
\newline

\noindent
\textbf{Paragraaf 3.4 Opdracht}
\\*Wat is de projectopdracht in precieze bewwordingen?
\\*Wat wordt wel en wat wordt niet gerekend tot de opdracht?
\\*Wat zijn de eisen en beperkingen die de opdrachtgever stelt aan tijd, geld, mensen of middelen?
\\*Wie is de opdrachtgever en wat zal hij bijdragen aan het realiseren van de opdracht?
\newline

\noindent
\textbf{Paragraaf 3.5 Risicofactoren}
\\*Welke risico's zijn er bij het behalen van het gewenste resultaat?
\\*Welke zijn het belangrijkste en hebben de hoogste prioriteit, en welke een lagere?
\\*Welke maatregelen worden er genomen?
\newline

\noindent
\textbf{Paragraaf 3.6 Succesfactoren}
\\*Wat zijn de succesfactoren die de opdrachtgever onderkend heeft?
\newline
\newline

Hoofdstuk 4: Projectfasering
\\*
\\*Dit hoofdstuk biedt een antwoord op de vragen:
\\*Op welke manier wordt het projectresultaat gerealiseerd?
\\*In welke fasen wordt het projectresultaat gerealiseerd?
\newline

\noindent
\textbf{Paragraaf 4.1 Inleiding}
\\*Welke aandachtsgebieden en welke fasen worden gebruikt?
\\*Hoe ziet de resultatenmatrix eruit?
\newline

\noindent
\textbf{Paragraaf 4.2 Aandachtsgebied 1}
\\*Welke activiteiten in welke fasen zijn hier nodig?
\\*Wat zijn de tussenresultaten die deze activiteiten opleveren?
\newline

\noindent
\textbf{Paragraaf 4.3 Aandachtsgebied 2}
\\*Welke activiteiten in welke fase zijn hier nodig?
\\*Wat zijn de tussenresultaten die deze activiteiten opleveren?
\newline

\noindent
\textbf{Paragraaf 4.4 Afhankelijkheden}
\\*Zijn er tussen de verschillende activiteiten relevante afhankelijkheden en zo ja, welke?
\newline
\newline

Hoofdstuk 5 Projectkader
\\*
\\*In dit hoofdstuk geeft u antwoord op de vragen 'wie', 'waarmee' en 'waar'.
\\*Binnen welk kader mag het project zich begeven om het eindresultaat te realiseren?
\newline

\noindent
\textbf{Paragraaf 5.1 Inleiding}
\\*Wat is het belang dat dit onderwerp, het projectkader, heeft voor het welslagen van het totale project?
\\*Wat zijn de voorwaarden die de projectmanager stelt, zowel naar het project (intern) als naar de omgeving (extern)?
\newline

\noindent
\textbf{Paragraaf 5.2 Projectorganisatie}
\\*Wat zijn de profielschetsen?
\\*Wat is het organogram en/of het verantwoordelijkheidsschema?
\newline

\noindent
\textbf{Paragraaf 5.3 Voorwaarden aan opdrachtgever}
\\*Wat zijn de voorwaarden die de opdrachtgever dient te realiseren?
\newline

\noindent
\textbf{Paragraaf 5.4 Voorwaarden aan derden}
\\*Wat zijn de eisen die gesteld worden aan derden?
\newline

\noindent
\textbf{Paragraaf 5.5 Projectcommunicatie}
\\*Welke formele communicatie is van belang voor het slagen van het project, zowel intern als extern?
\newline

\noindent
\textbf{Paragraaf 5.6 Faciliteiten en hulpmiddelen}
\\*Welke faciliteiten en hulpmiddelen zijn er nodig om het project te realiseren?
\newline

\noindent
\textbf{Paragraaf 5.7 Procedures en richtlijnen}
\\*Welke procedures en richtlijnen binnen en buiten het project zijn nodig om het project te laten slagen?
\newline
\newline

Hoofdstuk 6 Projectplanning
\\*
\\*Dit hoofdstuk geeft antwoord op de vragen:
\\*Wanneer wordt het eindresultaat opgeleverd?
\\*Wanneer worden de tussenresultaten opgeleverd?
\\*Welke mensen en middelen zijn wanneer nodig?
\\*Wat zijn de financiële consequenties?
\newline

\noindent
\textbf{Paragraaf 6.1 Normen en aannames}
\\*Welke normen, bijvoorbeeld wat betreft het ziektepercentage, worden gehanteerd bij de planning?
\newline

\noindent
\textbf{Paragraaf 6.2 Activiteitenplanning}
\\*Hoeveel tijd kost het om de activiteiten, zoals ze beschreven staan bij de projectfasering, uit te voeren?
\\*Wat zijn de afhankelijkheden tussen de activiteiten?
\newline

\noindent
\textbf{Paragraaf 6.3 Capaciteitsplanning}
\\*Welk verband is er tussen de tijd en de doorlooptijd van de activiteiten?
\\*Wie gaat welke activiteiten uitvoeren?
\\*Welke personen zijn er vanaf welke datum nodig binnen het project en wanneer kunnen zij uitstromen?
\\*Welke materialen en welk materieel zijn wanneer nodig?
\newline

\noindent
\textbf{Paragraaf 6.4 Mijlpalenplanning}
\\*Op welke data worden de mijlpalen bereikt?
\newline

\noindent
\textbf{Paragraaf 6.5 Financiële planning}
\\*Wat zijn de kosten van het project?
\end{document}

