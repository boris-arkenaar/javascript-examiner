% individueel verslag van de persoonlijke ervaringen en leermomenten

\section{Boris Arkenaar}
% - Finding the balance between
%   delivering the requirements of the product owner
%   and developing a solid foundation, writing good code,
%   writing comments and writing documentation.
% - Even more working together would be great.
% - Balance between prototyping and modeling/documenting beforehand.
From the start I thought this was a great project to be a part of.
Building an application that will actually be used
and will be valuable to both \glspl{student} and \glspl{tutor}.
Delving deeper into the \gls{js} language (I already knew quite well).
Experience the whole process from begining to end.
And work together on it with two other students, as a team.

When starting off I find it difficult to find the right direction,
to confine the project to what realy needs to be done.
Especially in the form of writing a project plan
and writing a paper on a domain research.
That's a lot of research and paper work
while not having a good grip on the impact of the project.
Writing the project plan and doing the research
is ofcourse ment to confine the project
and getting a good grip on the things that need to done.
But in retrospective it might seem a good option
to minimize the paperwork up front,
start developing a prototype as soon as possible,
also demonstrating it very early on to the product owner.
And then come back and refine the research and project plan,
while have a better understanding of what you are doing
and going to be doing.
This all generating a more iterative approach
between documentation and development.

Of course we applied an iterative approach for the development phase.
This went very well.
It gave us the incentive to produce a working version of the application
at a timely maner.
Demonstrating the application after each iteration
to receive feedback from the product owner.
And making ourselves aware of what we are doing
and how we are doing it.
The feedback of the product owner is very valueable,
however only focused on the end result.
With the requirements of the product owner in mind
it is difficult to keep enough time and effort available
for implementing a good foundation for all those requirements.
Especially for a starting with a new product in this case,
a good foundation is very important
to enable the teams after us to continue our path
and extend the application with new functionality.

Another factor making it difficult to produce well written code
and a good foundation
is that we started out writing a prototype.
We we're exploring the requirements and how to implement them,
the possibilities of the research
and what technologies to use.
All this making it very unsure where we would be in ten days.
And thus you write code to try things out,
try to implement a requirement,
see if you can use something from the research,
determine if a technologie can be adopted.
You don't write that code too well,
because you are not sure if you're even gonna use it in the end.
This all requires refactoring afterwards,
you need too keep room for that in your planning though.
The prototyping also reflects on commenting and documenting code
making diagrams to create a clear view of the architecture
and developing using a test driven approach.
Like I said I think it might have been a nice approach
to start prototyping very early on,
then incrementally create the architecture, requirements and documentation.
And at a certain point stop prototyping
and start developing according to a more structured, test driven approach,
to reduce the uncertainty
and start working to a clear goal and finished product.

\section{Bram Nieuwenhuize}
% - Developing software through a team effort X
% - Developing in a structured way X
% - New platforms (Node.js, Polymer, MongoDB) X
% - Tension research, development and deliver X
% - Importance of writing test code X
% - Combining study and work
% - Taking control X
My (learning) experiences in this project range from the new experience of 
combining study activity while working in the same field to learning to develop 
on new platforms.
To start with the latter, I had very little knowledge of 
JavaScript, Node.Js and no knowledge at all of Polymer and 
MongoDB. This was somewhat foreseen, as I had no 
professional developer experience when this project 
commenced. Developing in a team and with a
structured approach (instead of starting to code until it started to break),
was new for me as well. During the project, the importance of 
researching and planning became came more and more apparent. I am getting better
at neglecting my primal instinct to start coding right away. The relation 
between the amount of preparation and planning and the eventual result prove 
this is something I have to keep working on. I learned that for me it's often
good practice to postpone the coding itself as long as possible, and when I will
eventually get to that, start with the test suites, thus developing in a Test 
First Development way. The ease of writing these test suites for the Node.js
code and the utility they had is an experience I will take into account in my
future development projects.

When overseeing the whole project, I experienced some tension between the
project, the project structure (mainly the research aspects) and my own agenda.
In the beginning of the project my orientation was quite research minded, in the
sense I felt an urge to research beyond the current project, and learn about 
feedback in general, didactics in distance learning, online education. Luckily
I was able dive into this matter in the domain research. At the same time, I
noticed some skepticism about the need for such research. Later on in the 
project, it became clear to me this project is primarily about creating the 
tool. The turn to a more pragmatic approach and eventually even to a quasi 
production ready tool, is a nice experience after all. It stressed our 
abilities to adjust, a competence of great value in a fast shifting field like 
that of software engineering. My initially intended reticent attitude (not taking control all the time), only succeeded sparingly during the moments we had to shift approach. This is something I want to keep working on.

After the project started I attended to my first job as a software developer. 
Before I was working in another field, which made the separation between work 
and study fairly clear. With this new job, the demarcation vanished. I notice 
it's sometimes hard to direct my attention towards work when intended and vice 
versa. 


\section{Ronald Kluft}
% - working in a team
% - Scrum
% - Combining study family and work
% - knowledge JS / Node.js / Gulp / 
% - writing article
% - searching literature

At the beginning of this project, I had little knowledge of the \gls{js}
language. I was the only student in our team that hadn't followed the course
\gls{wac}. So, to have a better understanding, I have worked my way through
this course.
As we were programming the \gls{examiner} in \gls{js-code} it was even a necessity.

Also, because I didn't have experience in writing an academic article, I
made it a goal to gain this competence.
Another competence to gain was searching for useful literature.
For both there is a mini-course offered by the Open University, and I have
used them to reach my goals.
It was the first time I came across these , but I had hoped that there were
other times I was supposed to write an article, so that I have had more
experience in doing so.

I sometimes had trouble finding the right balance between family, work and study.
As a father of two young children, it is hard to be as flexible in making
appointments as I wanted to be.
It sometimes was impossible to attend a meeting face-to-face, because I couldn't
arrange a child-minder to look after my children.
Luckily, there was always the possibility to contact the other team members by
Google hangouts.

Working in a team wasn't a difficulty for me. In my day-to-day job, I work in
teams as well, as a part of Scrum. Face-to-face communication is also a skill I
am familiar with.
But it is sometimes difficult to make an appointment, and then an email was the
way to communicate. 
The team meetings on location were almost always very constructive.
Though it takes effort to travel, it makes it so much easier to come to an
agreement.

The project itself went well.
It was clear that it wasn't only about writing a software program, but that we
had to manage the complete project.
It already started by writing an application for the project.
The fact that we as a team had to do every aspect of the project was also a
learning experience. 
At work a number of things are done by other teams, for instance testing.
Now this was also a skill I had to master.

At the end, I must admit that is was a very educational time. I have learned a lot of tools and techniques that I hadn't heard of before.
For instance, Node.js is a very powerful tool to work with.
Latex is also a powerful tool, but whether I am going to use it in the future is
not likely, because there are other editors which use are less complex.
I have enhanced my skills with the competences of academic writing, literature
study, working in a team and managing a project.

The experiences I have learned during this project will certainly be useful in
the rest of my career. Although I haven't developed a preference for \gls{js},
the structured approach of the project will be valuable.
