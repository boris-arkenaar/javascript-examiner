% individueel verslag van de persoonlijke ervaringen en leermomenten

\section{Boris Arkenaar}
% - Finding the balance between
%   delivering the requirements of the product owner
%   and developing a solid foundation, writing good code,
%   writing comments and writing documentation.

\section{Bram Nieuwenhuize}
% - Developing software through a team effort X
% - Developing in a structured way X
% - New platforms (Node.js, Polymer, MongoDB) X
% - Tension research, development and deliver X
% - Importance of writing test code X
% - Combining study and work
% - Taking control X
My (learning) experiences in this project range from the new experience of 
combining study activity while working in the same field to learning to develop 
on new platforms.
To start with the latter, I had very little knowledge of 
JavaScript, Node.Js and no knowledge at all of Polymer and 
MongoDB. This was somewhat foreseen, as I had no 
professional developer experience when this project 
commenced. Developing in a team and with a
structured approach (instead of starting to code until it started to break),
was new for me as well. During the project, the importance of 
researching and planning became came more and more apparent. I am getting better
at neglecting my primal instinct to start coding right away. The relation 
between the amount of preparation and planning and the eventual result proof 
this is something I have to keep working on. I learned that for me it's often
good practice to postpone the coding itself as long as possible, and when I will
eventually get to that, start with the test suites, thus developing in a Test 
First Development way. The ease of writing these test suites for the Node.js
code and the utility they had is an experience I will take into account in my
future development projects.

When overseeing the whole project, I experienced some tension between the
project, the project structure (mainly the research aspects) and my own agenda.
In the beginning of the project my orientation was quite research minded, in the
sense I felt an urge to research beyond the current project, and learn about 
feedback in general, didactics in distance learning, online education. Luckily
I was able dive into this matter in the domain research. At the same time, I
noticed some skepticism about the need for such research. Later on in the 
project, it became clear to me this project is primarily about creating the 
tool. The turn to a more pragmatic approach and eventually even to a quasi 
production ready tool, is a nice experience after all. It stressed our 
abilities to adjust, a competence of great value in a fast shifting field like 
that of software engineering. My initially intended reticent attitude (not taking control all the time), only succeeded sparingly during the moments we had to shift approach. This is something I want to keep working on.

After the project started I attended to my first job as a software developer. 
Before I was working in another field, which made the separation between work 
and study fairly clear. With this new job, the demarcation vanished. I notice 
it's sometimes hard to direct my attention towards work when intended and vice 
versa. 


