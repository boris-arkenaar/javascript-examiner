% introductie, plusminus 3 A4 pagina's, gezamenlijk
% (mogelijke onderdelen: context, vraagstelling etc)
\chapter{Introduction}


\section{ABI}
% a general part about ABI
The Afstudeerproject Bachelor Informatica (ABI) is the last part of the Bachelor curriculum.
In the introduction meeting of the ABI, we were attracted to the \gls{examiner} project.
The initial goal of this project was to create a prototype to automatically examine
software code consint of \gls{js} an HTML.

The project is not only about creating a product, but also a chance to work in a team, write an academic article/thesis, and to work in a team. This involves planning, communication etc.

The product-owner, Dr.ir Harrie Passier, is a tutor of the \gls{wac} course.


% - introduction/introduction}

% - Research
\section{Research}
% -  research/research
% -    Domain research
% -      Introduction
% -       research/domain-research-introduction
% -       research/domain-research
% -    Research Context}
% -       research/research-context

% - Product
% -  product/product
% -    Requirements
% -      product/requirements
% -    Architecture
% -      product/architecture
% -    Checks
% -      product/checks

% -    Frontend
\section{Frontend}
The frontend section explains what modules have been chosen and why.
It handles the UI framework Polymer (with a part about Web components and
Isolation), and the module Code Mirror.
It is used to give
the user an easy to use interface to submit their code.

% -    Backend
\section{Backend}
This part is about the server side logic. It gives a detailed analysis of the
package manager and the individual packages that are used.
in the development. It also is about the structure of the backend.
It talks about the routing, authentication and persistence.

% -    Maintainability
% -      product/maintainability}
 
% - Process
% -  process/process
% -    Process Report
% -      process/process-report
% -    Team Reflection
% -      process/team-reflection
% -    Experience
% -      process/experience

% - conclusion/conclusion

