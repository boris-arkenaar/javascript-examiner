% introductie, plusminus 3 A4 pagina's, gezamenlijk
% (mogelijke onderdelen: context, vraagstelling etc)
\chapter{Introduction}


% - tools
% -- Github
% -- Latex
% -- Google Hangouts
% -- Skype
% -- Subversion
% -- Node.js
% -- 
% --
% --
% - server
% -- subversion
% -- production

\section{Context} 

In the introduction meeting of the ABI, we were attached to a project that aims 
at creating a prototype that enables automatic
examination of \gls{js} and HTML. The initial description of the project stated 
that it was the goal to create a prototype for a software tool to 
check the validity of \gls{js-code}. This tool should initially be able to
facilitate the tutors and student of a course of the Open University.
The course \gls{wac} is a part of the curriculum in the Computer Science faculty,
that aims at providing students an introductory course about web programming.
HTML5, CSS and mostly \gls{js} are the main parts of this course.
The product owner, Dr.ir Harrie Passier, is a tutor of the aforementioned course.
The expectation of the product owner was to deliver a prototype to examine 
\gls{js-code}. Along the way, it became clear that the expectations
shifted from a prototype towards a working environment, which was ready to be 
used in the upcoming session of the \gls{wac} course.


The project is not only about creating the tool, but also a means to 
improve academic research skills and to work in a team. This introduction will 
outline the structure of the thesis, by introducing each part and compelling the 
role of this part within the project. There are three main parts: Research, 
Product and Process. The Research part bundles all academic research artifacts, 
whereas the Product related chapters are merely a practical outline of the 
developed tool, which will be referred to as \gls{examiner} from now on. The 
process of the project is described in the last part.

% - Research
\section{Research}
% -  research/research
% -    Domain research
% -      Introduction
This part is about the context and domains wherein this project takes place, 
shaped by the produced research artifacts.
It is about the actors and their eventual purposes with the \gls{examiner}. One
of the domain studies researches the domain of Feedback from several distinct
perspectives. A didactic method on learning a programming language and
analysis of the programming language itself, are researched too. The research 
context section places the product in perspective. Related studies and efforts 
have been analyzed. A closely related and matured academic project has been 
analyzed more closely through a consult with one of the main stakeholders.
% -       research/domain-research-introduction
% -       research/domain-research
% -    Research Context}
% -       research/research-context
The research resulted into a set of requirements, which enabled the transition
from research to the actual implementation. These extracted requirements have
been divided in two ways, namely the separation between functional and 
non-functional, and the separation of requirements that are met during this
project, and possible requirements for subsequent projects.

% - Product
\section{Product}
Based on the requirements, it became possible to grasp the general conditions 
to which the tool must apply. Based on these conditions, a general architecture
was deduced. The related chapter clarifies the architectural structure, 
containing the relation between domain, \glspl{check}, RESTful server and frontend.
From this overview, the several components are described to great detail, in
order to facilitate succeeding contributions like subsequent projects.

% -  product/product
% -    Requirements
% -      product/requirements
% -    Architecture
% -      product/architecture
% -    Checks
% -      product/checks

% -    Frontend
% - \section{Frontend}
The frontend section explains what modules have been chosen and why.
It describes the UI framework Polymer (with a section about Web components and
Isolation), and the module Code Mirror. This module enables an easy to use 
interface for code submission and providing feedback.
% -    Backend
% - \section{Backend}
The backend is about the server side logic. It gives a detailed analysis of the
package manager and the individual packages that are used
in the development. It also explains the structure of the backend.
%   checks.
\Glspl{check} are separated modules which are called to make a statement
about the validity and structure of the submitted \gls{js-code}.
% -    Maintainability
% -      product/maintainability}
The maintainability part discusses the possibility to extend the tool with other 
modules. One implication of extendibility is a demand for maintainability. 
Through explicit definition and assigning high priority to this requirement from 
the start, a detailed description of what to keep in mind when 
preserving this maintainability could be concluded.


% TODO FIX THIS transfer to Process section or to product section, discuss tomorrow:

% \section{Tools}

% Right from the start there was an idea about which tools we would start to use, 
% to create the examiner. We were free to write it in any programming language, and
% at first we were also thinking about Haskell.			
% But as it was supposed to be used with \gls{js}, and the knowledge within our
% team was better 
% , we decided to use a platform based on \gls{js} for the construction of the tool. 
% at first we were also thinking about Haskell.
% But as it was supposed to be used with \gls{js}, and because the knowledge about \gls{js} within our
% team was better % Dit deel van de zin is onduidelijk, wat bedoel je? (Oizor)
% % Ik denk dat wat ik er nu van gemaakt heb (Slotkenov).
% , we decided to use \gls{js} as well for the construction of the tool.

% % Probeer in een verhaal te schrijven, dus niet tool 1, en tool2 en tool3 etc, maar
% % Meer de verhouding tot elkaar; hoe het samen een geheel vormt.
% % Node.js is het platform.
% Along the way we have discovered a number of tools that helped us in creating the various parts of the project.
% There is Node.js, which offers an environment to run \gls{js-code}.
% % Onderstaande zin loopt niet.
% Npm, part of Node.js, which is a package manager, so you don't have to install
% all the dependent packages by hand.

% Gulp is used to make a server and restart after every change in a text file.
% Latex to create the reports and thesis.

% % Probeer hier de lezer te motiveren om deze stukken te gaan lezen, wat staat er in,
% % waarom is het interessant voor het onderzoek en wat zijn de belangrijkste bevindingen.

% END TODO

% - Process
\section{Process}

The process part holds a discussion about the choices we made
during the process of developing the \gls{examiner}. The choices regarding 
modeling and development decisions are discussed, enlightening the choices made
on research, product and the process itself. The section also holds a description
of how we have experienced this project, both in a team reflection as an 
individual experience report.

% -  process/process
% -    Process Report
% -      process/process-report
% -    Team Reflection
% -      process/team-reflection
% -    Experience
% -      process/experience

% - conclusion/conclusion

Although the \gls{examiner} is far from mature, we do believe this project led
to a solid base for succeeding teams. Throughout the document we have made
suggestions for possible future adjustments and extensions.
