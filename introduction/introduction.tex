% introductie, plusminus 3 A4 pagina's, gezamenlijk
% (mogelijke onderdelen: context, vraagstelling etc)
\chapter{Introduction}

% - tools
% - server


\section{ABI}
% a general part about ABI

The Afstudeerproject Bachelor Informatica (ABI) is the last part of the Bachelor curriculum.
In the introduction meeting of the ABI, we were attracted to the \gls{examiner} project.
The initial description of the project stated that it was the goal to create,
at least making a start, of a software tool to check the validity
of \gls{js-code}.

It is not only about creating a product, but also a chance to work in a team, write an academic article/thesis, and to work in a team.
This involves planning, communication and etc.


% an introduction of the team
\section{Team}
At the beginning, already a face-to-face meeting was planned, to work out the initial ideas we had.
A positive coincidence was that we are geographically not far apart.


\section{Javascript-examiner}
The course \gls{wac} is a part of the curriculum of the Open University to give
a student an introductory course about web programming.
HTML5, CSS and \gls{js} are part of this course.
The \gls{examiner} is a tool to be used as part of the chapters about the JavaScript language.
It is supposed to validate a submission of a given exercise of a student.


\section{Product-owner}

The product-owner himself is a tutor of the aforementioned course.
Together with an other tutor, they have supplied structural information about
what we were expected to create.


\section{Goal}


\section{Expectation}


\section{Tools}

Right from the start there was an idea about which tools we would start to use, to create the examiner. We were free to write it in any programming language, and
at first we were also thinking about Haskell.
But as it was supposed to be used with \gls{js}, and the knowledge within our
team was better, we decided to use \gls{js} as well for the construction of the tool. 

Along the way we have discovered a number of tools that helped us in creating the various parts of the project.
There is Node.js, which offers an environment to run \gls{js-code}.
Npm, part of Node.js, which is a package manager, so you don't have to install
all the dependent packages by hand.
Gulp is used to make a server and restart after every change in a text file.
Latex to create the reports and thesis.


\section{Domain research}
Part of the project was also a personal domain research. At first, we thought up
three domains to divide between us.
Semantics, Feedback and tools.
After the research of the tools, it became clear that this wasn't a good domain
to research.
The research wasn't obsolete, but it had not sufficient academic content to be
part of the project.
There was use for it later in the actual development phase of the \gls{examiner}.