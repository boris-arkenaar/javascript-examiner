\documentclass{article}

\begin{document}

\title{Requirements}
\author{Ronald Kluft \and Bram Nieuwenhuize \and Boris Arkenaar}
\date{\today\\v1.0}
\maketitle

\section{Requirements for current development}
\subsection{Functional Requirements}
\subsubsection{Interaction}
\begin{description}
  \item[Exercise creation] The tutor should be able to create an exercise
    consisting of a description, a solution and one or more unit tests.
  \item[Exercise modification] The tutor should be able to modify or delete an
    exercise.
  \item[Exercise description retrieval] The student should be able to retrieve
    the description of an exercise.
  \item[Code submission] The student should be able to submit the JavaScript 
    code solution he created.
  \item[Feedback] The student should receive feedback on the submitted 
    JavaScript code solution. 
\end{description}

\subsubsection{Examination}
\begin{description}
  \item[Unit testing] The JavaScript-examiner should be able to execute unit 
    tests for given JavaScript code.
  \item[Code layout check] The JavaScript-examiner should be able to check if 
    the submitted JavaScript code adheres to a defined code layout and style.
  \item[Maintainability metrics calculation] The JavaScript-examiner should be 
    able to calculate the cyclomatic complexity, Halstead complexity and the 
    logical lines of code (LLOC) from the submitted JavaScript code.
  \item[Results comparison] The JavaScript-examiner should be able to compare
    the results from the calculations of the student's submitted JavaScript
    code with the results of the tutor's Javascript code. The
    JavaScript-examiner should conclude the comparison with some advice to the
    student.
  \item[Redundancy] The Javascript-examiner should be able to check for
    redundancy.
  \item[Constructors] The JavaScript-examiner should be able to check for
    proper constructors.
  \item[Variable declaration] The JavaScript-examiner should be able to check
    for proper declaration of variables.
\end{description}

\subsubsection{Feedback}
\begin{description}
  \item[Useful] The JavaScript-examiner should provide feedback in a way
    the student is supported to improve his skills.
  \item{Motivational} The JavaScript-examiner should provide feedback in such a
    manner that it motivates the student to continue.
  \item[Elegance] The JavaScript-examiner should return well useful and written 
    feedback, or at least encourage the tutor to write well written feedback.
  \item{Format} The format of the returned feedback should be easily printable.
\end{description}

\subsection{Non-functional Requirements}
\begin{description}
  \item[Anonymity] The student should be anonymous while using the
    JavaScript-examiner: the submitted JavaScript code may not be traced back
    to a student.
  \item[Plagiarism detection] The JavaScript-examiner should be able to detect
    plagiarism.
  \item[Feedback on platform and exercises] Students should be able to provide
    feedback on JavaScript-examiner in general and exercises in particular.
  \item[Security in executing submitted code] Sufficient measures should be
    taken to guarantee the integrity of the system while executing submitted
    code.
  \item[Extensible exercises] The JavaScript-examiner should be designed in a
    way it's possible to extend the scope of exersises and even add support for
    other programming languages in the future. 
  \item[Store progress] The JavaScript-examiner should help the student keep
    track on progress.
\end{description}


\section{Requirements for future development}
\subsection{Functional Requirements}
\begin{description}
  \item[Relative execution time calculation] The JavaScript-examiner should be 
    able to calculate the relative execution time for a given code.
  \item[Language constructs detection] The JavaScript-examiner should be able 
    to detect, and point out to the student, when an important language 
    construct is missing that is adviced to use in the exercise in question.
  \item[Feedback among students] The JavaScript-examiner should enable feedback
    among each other to help each other with the exercises. 
  \item[Student test case submission] The students should be able to write and
    submit test cases for exercises, which the JavaScript-examiner can examine
    and add to the test suite of the particular exercise.
\end{description}
\subsection{Non-functional Requirements}

\end{document}
