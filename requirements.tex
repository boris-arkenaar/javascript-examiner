\documentclass{article}

\begin{document}

\title{Requirements}
\author{Ronald Kluft \and Bram Nieuwenhuize \and Boris Arkenaar}
\date{\today\\v1.0}
\maketitle

\section{Requirements for current development}

\subsection{Interaction}
\begin{description}
  \item[Exercise creation] The tutor should be able to create an exercise
    consisting of a description, a solution and one or more unit tests.
  \item[Code submission] The student should be able to submit a code he
    created as a solution to an exercise.
  \item[Submission response] The student should be able to receive a response
    to a submitted solution. Telling him if the solution works and how the
    solution might be improved.
\end{description}

\subsection{System}
\begin{description}
  \item[Unit testing] The system should be able to execute unit tests for a
    given code.
  \item[Code layout check] The system shoul be able to check if a given code
    adheres to a defined code layout.
  \item[Maintainability metrics calculation] The system should be able to
    calculate the cyclomatic complexity, Halstead complexity and the logial
    lines of code (LLOC) from a given code.
  \item[Results comparison] The system should be able to compare the results
    from the calculations of a student's code with the results of the tutor's
    code. The system should conclude the comparison with some advice to the
    student.
\end{description}

\section{Requirements for future development}
\begin{description}
  \item[Relative execution time calculation] The system should be able to
    calculate the relative execution time for a given code.
  \item[Language constructs detection] The system should be able to detect, and
    point out to the student, when an important language construct is missing
    that is adviced to use in the exercise in question.
\end{description}

\end{document}
